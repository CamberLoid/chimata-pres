% 01-intro.tex
% about 2-3mins
\section{课题综述}

\subsection{选题背景和研究意义}

\begin{frame}
    \frametitle{选题背景}

    % 背景包括
    \begin{itemize}
        \item 移动支付随着电子商务的兴起而被推广:
        \begin{itemize}
            \item 买卖双方在线下商定好金额后,通过某一支付服务提供商进行转账;
            \item 依托于云计算服务
        \end{itemize}
        \item 受到许多安全方面的挑战:
        \begin{itemize}
            \item 外部:攻击者通过各种方式获得密钥并解密信息,或是“先存储后解密”攻击;
            \item 内部:监守自盗,在未经用户许可下访问信息并非法构建用户画像;
        \end{itemize}
        \item 泄露后果:推断用户其他各种敏感信息
    \end{itemize}

\end{frame}

\begin{frame}
    \frametitle{研究意义}

    本文提出了一种金额密态的,适用于移动支付场景的交易方案,实现了用户之间的交易金额只由交易双方知道而不泄露给第三方。具体而言:
    \begin{itemize}
        \item 借助 CKKS 方案所基于的困难问题,保证交易金额的隐私性;
        \item 使用 CKKS 方案的同态特性,实现了对用户密文账户余额的同态操作以及手续费的计算,确保了可交易性;
        \item 本文的研究有助于拓展和提出更完备、功能更强的交易隐私保护方案,也有助于提高用户对移动支付的信任度。
    \end{itemize}

\end{frame}

\subsection{研究现状}

\begin{frame}
    \frametitle{相关工作}

    \begin{itemize}
        \item 全同态加密
        \begin{itemize}
            \item Craig G. 首次提出整数的全同态加密方案,包括自举手段等
            \item Cheon 等人在 2017 年提出支持浮点数加解密的同态加密方案 HEAAN,并被后人学者命名为 CKKS,通过自举达到全同态性。
        \end{itemize}
        \item 隐私交易方案 
        \begin{itemize}
            \item 2018 年,一个结合了区块链和同态加密的电子医疗记录隐私保护方案被提出,实现了保险公司等第三方在无法获取客户明文医疗记录的情况下,仍然可以判断是否理赔的功能;
            \item 2020 年,姜轶涵等人在他的文献里运用了同态加密、零知识证明和数字签名等密码学手段,提出了一个密态的可审计交易方案。
        \end{itemize}
    \end{itemize}

\end{frame}

\subsection{主要贡献和创新}

\begin{frame}
    \frametitle{主要贡献和创新}

    \begin{itemize}
        \item 对现有的交易隐私保护方法和全同态加密算法进行调研,提出了一个基于同态加密的用户交易金额隐私保护交易方案;
        \item 使用 CKKS 方案对浮点数的支持的特性,确保了交易的隐私性和可交易性,包括浮点数手续费的计算,以及账户密文余额同态更新等;
        \item 使用密钥交换确保了交易的正确性;使用 ECDSA 算法确保了不可伪造性;
        \item 使用 Golang 编写了一个简单的库实现,包括客户端和服务端逻辑、一个简单的服务器 CLI 程序,以及相关单元测试;
        \item 测试表明,方案在密码学方面具有良好的时间开销和可以接受的空间开销。
    \end{itemize}

\end{frame}

\subsection{论文结构安排}
\begin{frame}
    \frametitle{论文结构安排}

    \begin{enumerate}
        \item 绪论
        \item 基础知识
        \item 支付方案设计
        \item 具体的代码设计
        \item 实验评估
        \item 总结和展望
    \end{enumerate}

\end{frame}

